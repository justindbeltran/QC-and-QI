\documentclass[12pt]{exam}
\usepackage[
  letterpaper,
  margin=0.3in,
  includehead,    % include the header area in the 0.25in margin
  includefoot     % include the footer area in the 0.25in margin
]{geometry}
\usepackage[shortlabels]{enumitem}
\setlist[enumerate,1]{label=(\alph*)} %makes list alphabetic

\usepackage{physics}
\usepackage{amsfonts}
\newcommand{\C}{\mathbb{C}}
\usepackage{pgfplots} % For sketching graphs
\pgfplotsset{compat=1.18}


\usepackage{caption}

\usepackage[hidelinks]{hyperref}

\makeatletter
  % Save the original \section into \EXAM@orig@section
  \let\EXAM@orig@section\section
  \renewcommand{\section}[1]{
    \refstepcounter{section}
    \EXAM@orig@section*{#1~\thesection}
    
    \addcontentsline{toc}{section}{#1~\thesection}
  }
\makeatother

\usepackage{etoolbox}

\newcounter{exercise}[section]
\renewcommand{\theexercise}{\thesection.\arabic{exercise}}
\newcounter{problem}[section]
\renewcommand{\theproblem}{\thesection.\arabic{problem}}

\makeatletter
  \let\EXAM@orig@subsection\subsection

  %Redefine \subsection itself:
  \renewcommand{\subsection}[1]{
    \ifstrequal{#1}{Exercise}{
      \refstepcounter{exercise}
      \EXAM@orig@subsection*{#1~\theexercise}
      \addcontentsline{toc}{subsection}{#1~\theexercise}
    }{
      \ifstrequal{#1}{Problem}{
        \refstepcounter{problem}
        \EXAM@orig@subsection*{#1~\theproblem}
        \addcontentsline{toc}{subsection}{#1~\theproblem}
      }{
        \refstepcounter{subsection}%
        \EXAM@orig@subsection*{#1~\thesubsection}%
        \addcontentsline{toc}{subsection}{#1~\thesubsection}
      }
    }
  }
\makeatother

\usepackage{hanging}


\title{%
  \centering
  Quantum Computation and Quantum Information\\
  Solution Manual\\
  \small
  For Michael A. Nielsen and Isaac L. Chuang’s 10th Anniversary Edition
}

\author{By Justin Beltran}
\date{May 2025}

\begin{document}

\maketitle
\firstpagefooter{}{}{By Justin Beltran}
\runningfooter{}{}{By Justin Beltran}
\printanswers



\section{Chapter}

\subsection{Exercise}
\textbf{(Probabilistic classical algorithm)} Suppose that the problem is not to distinguish between the constant and balanced functions \textit{with certainty}, but rather, with some probability of error $\epsilon < 1/2$. What is the performance of the best classical algorithm for this problem?

\begin{solution}
    The best classical algorithm for Deutsch's problem with 100\% success would require $2^{n-1}+1$ queries because Alice may receive $2^{n-1}$ 0's before receiving a 1. The best probabalistic classical algorithm for the same problem with $\epsilon < 1/2$ would be 3 queries. The first query establishes a baseline and the second has probability 1/2 of matching the first but 3 queries would have probability 1/4. Generalizing this for $q$ queries with probability of error $\epsilon$ is $$\epsilon < \frac{1}{2^{q-1}} = 2^{-q+1} \quad \implies \quad \log\frac{1}{\epsilon} + 1 < q \quad \implies \quad O\big(\log\frac{1}{\epsilon}\big)$$
\end{solution}







\end{document}
